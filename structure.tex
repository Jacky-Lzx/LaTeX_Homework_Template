%%%%%%%%%%%%%%%%%%%%%%%%%%%%%%%%%%%%%%%%%
% Cleese Assignment (For Students)
% LaTeX Template
% Version 4.0 (10/06/2025)
%
% This template originates from: http://www.LaTeXTemplates.com
%
% Author:
%   Original: Vel (vel@LaTeXTemplates.com)
%   Modified Jacky.Li - Added some custom commands, modified some displays and added code listing support
%
% License:
% CC BY-NC-SA 3.0 (http://creativecommons.org/licenses/by-nc-sa/3.0/)
%%%%%%%%%%%%%%%%%%%%%%%%%%%%%%%%%%%%%%%%%

\usepackage{lastpage} % Required to determine the last page number for the footer
\usepackage{graphicx} % Required to insert images

% \setlength\parindent{0pt} % Removes all indentation from paragraphs

\usepackage[most]{tcolorbox} % Required for boxes that split across pages
\usepackage{booktabs} % Required for better horizontal rules in tables
\usepackage{etoolbox} % Required for if statements

% \usepackage{ulem} % Cannot use with algorithm2e used in `code-style.tex`

% \usepackage{setspace}
% \renewcommand{\baselinestretch}{1.0}

% \usepackage{subcaption}

\usepackage{float}
\usepackage{amsmath, amssymb, amsthm}
\usepackage{bm}
\usepackage{xcolor}
\usepackage{cprotect}

\usepackage{xurl}
\usepackage[colorlinks=true,linkcolor=blue, citecolor=violet]{hyperref}

% \usepackage{bookmark}

\usepackage[inline]{enumitem}

\newlist{refer-enum}{enumerate}{1}
\setlist[refer-enum]{label=[\arabic*].,before=\raggedright}
\newenvironment{reference}{%
  \vspace{1em}
  \par \textbf{Reference}:
  \begin{refer-enum}
  }{%
  \end{refer-enum}
}

\usepackage{ifthen}

\usepackage{siunitx}

\renewcommand{\d}{\: \mathrm{d} }
\newcommand{\e}{\mathrm{e}}

\newcommand{\ds}{\displaystyle}

% Used for time complexity
\newcommand{\OO}{\mathcal{O}}
\newcommand{\TT}{\Theta}
\newcommand{\RR}{\mathbb{R} }
\newcommand{\abs}[1]{|#1|}
\newcommand{\Abs}[1]{\left|#1\right|}
\newcommand{\norm}[1]{\|#1\|}
\newcommand{\normone}[1]{\|#1\|_1}
\newcommand{\normtwo}[1]{\|#1\|_2}
\newcommand{\norminf}[1]{\|#1\|_\infty}

\newcommand{\TODO}[1]{{\bfseries \textcolor{red}{TODO\@: #1}}}
\newcommand{\modify}[2]{\textcolor{gray}{\sout{#1}} \textcolor{blue}{#2}}

%----------------------------------------------------------------------------------------
%  FONT
%----------------------------------------------------------------------------------------
% \usepackage[utf8]{inputenc} % Required for inputting international characters
\usepackage[T1]{fontenc} % Output font encoding for international characters

\newcommand*{\defaultfont}{\fontfamily{lmr}\selectfont}

\usepackage[sfdefault, light]{roboto}

%----------------------------------------------------------------------------------------
%  MARGINS
%----------------------------------------------------------------------------------------
\usepackage{geometry} % Required for adjusting page dimensions and margins
\geometry{
  paper=a4paper, % Change to letterpaper for US letter
  top=3cm, % Top margin
  bottom=3cm, % Bottom margin
  left=2.5cm, % Left margin
  right=2.5cm, % Right margin
  headheight=14pt, % Header height
  footskip=1.4cm, % Space from the bottom margin to the baseline of the footer
  headsep=1.2cm, % Space from the top margin to the baseline of the header
  % showframe, % Uncomment to show how the type block is set on the page
}

%----------------------------------------------------------------------------------------
%  HEADERS AND FOOTERS
%----------------------------------------------------------------------------------------
\usepackage{fancyhdr} % Required for customizing headers and footers
\pagestyle{fancy} % Enable custom headers and footers

% Left header
% Output the instructor in brackets if one was set
\lhead{\small\assignmentClass\ifdef{\assignmentClassInstructor}{\ (\assignmentClassInstructor):}{}\ \assignmentTitle}
% Centre header
\chead{}
% Right header
% Output the author name if one was set, otherwise the due date if that was set
\rhead{\small\ifdef{\assignmentAuthorName}{\assignmentAuthorName}{\ifdef{\assignmentDueDate}{Due\ \assignmentDueDate}{}}}

\lfoot{} % Left footer
\cfoot{\small Page\ \thepage\ of\ \pageref{LastPage}} % Centre footer
\rfoot{} % Right footer

\renewcommand\headrulewidth{0.5pt} % Thickness of the header rule

%----------------------------------------------------------------------------------------
%  MODIFY SECTION STYLES
%----------------------------------------------------------------------------------------
\usepackage{titlesec} % Required for modifying sections

%------------------------------------------------
% Section
% \titleformat{<command>}[<shape>]{<format>}{<label>}{<sep>}{<before-code>}[<after-code>]
\titleformat{\section} %
[block] %
{\Large\bfseries} %
{\assignmentQuestionName~\thesection} %
{3pt} %
{\mdseries\itshape} % ChkTeX 6

% Spacing around section titles, the order is: left, before and after
% \titlespacing{<command>}{<left>}{<before-sep>}{<after-sep>}[<right-sep>]
\titlespacing{\section} %
{0pt} %
{0.5\baselineskip} %
{0.5\baselineskip}

%------------------------------------------------
% Subsection
\titleformat{\subsection} %
[block] %
{\itshape \bfseries} % ChkTeX 6
{\arabic{section}.\arabic{subsection}} %
{4pt} %
{}

\titlespacing{\subsection} %
{1.5em} %
{0.5\baselineskip} %
{0.5\baselineskip}

\renewcommand\thesubsection{(\alph{subsection})}

%------------------------------------------------
% subsubsection
\titleformat{\subsubsection}
[runin]
{\itshape \bfseries} % ChkTeX 6
{\arabic{section}.\arabic{subsection}.\arabic{subsubsection}}
{4pt}
{}

\titlespacing{\subsubsection}
{3em}
{0.5\baselineskip}
{0.5\baselineskip}

\renewcommand\thesubsubsection{\arabic{subsubsection}.}

%----------------------------------------------------------------------------------------
%  CUSTOM QUESTION COMMANDS/ENVIRONMENTS
%----------------------------------------------------------------------------------------
% Environment to be used for each question in the assignment
\newenvironment{question}[2][]{
  \pdfbookmark[0]{\assignmentQuestionName{} \the\numexpr\value{section}+1\relax}{\assignmentQuestionName{} \the\numexpr\value{section}+1\relax}
  \hypersetup{bookmarksdepth=-1}
  \ifthenelse{\equal{#1}{}}{
    \ifthenelse{\equal{#2}{}}{
    \section{}}{
    \section{--- #2}} %
    % \everymath={\displaystyle}
  }{
    \ifthenelse{\equal{#2}{}}{
    \section{{\Large\bfseries(#1)}}}{
    \section{{\Large\bfseries(#1)} --- #2}} %
  }
  \hypersetup{bookmarksdepth} %back
  % Set the left footer to state the question continues on the next page, this is reset to nothing if it doesn't (below)
  \lfoot{\small\itshape\assignmentQuestionName~\thesection~continued on next page\ldots} % ChkTeX 6
}{
  % Reset the left footer to nothing if the current question does not continue on the next page
  \lfoot{}
}

%------------------------------------------------
% Environment for subquestions
\newenvironment{subquestion}[2][]{
  \pdfbookmark[1]{\the\value{section}.\the\numexpr\value{subsection}+1\relax}{\the\value{section}.\the\numexpr\value{subsection}+1\relax}
  \hypersetup{bookmarksdepth=-1}
  \ifthenelse{\equal{#1}{}}{
    \ifthenelse{\equal{#2}{}}{
    \subsection{}}{
    \subsection{#2}} %
  }{
    \ifthenelse{\equal{#2}{}}{
    \subsection{(#1)}}{
    \subsection{(#1)\ #2}} %
  }
  \hypersetup{bookmarksdepth} % back
}{
}

% Environment for subsubquestions, takes 1 argument - the score
\newenvironment{subsubquestion}[2][]{
  \pdfbookmark[2]{\the\value{section}.\the\value{subsection}.\the\numexpr\value{subsubsection}+1\relax}{\the\value{section}.\the\value{subsection}.\the\numexpr\value{subsubsection}+1\relax}
  \hypersetup{bookmarksdepth=-1}
  \ifthenelse{\equal{#1}{}}{
    \ifthenelse{\equal{#2}{}}{
    \subsubsection{}}{
    \subsubsection{#2}}%
  }{
    \ifthenelse{\equal{#2}{}}{
    \subsubsection{(#1)}}{
    \subsubsection{(#1)\ #2}}%
  }
  \hypersetup{bookmarksdepth} % back
}{ %
}

\newenvironment{answer}{
  \begin{tcolorbox}[%
      breakable,%
      enhanced,%
      title=Answer,%
      title filled,%
      colframe=red!50!white,%
      colbacktitle=red!50!white,%
      attach boxed title to top left = {yshift = -3mm, xshift = 5mm},%
      colback=red!1%
    ]
    % \fontfamily{Xcharter}\selectfont
    \vspace{0.5em}
    % \fontfamily{cmr}\selectfont
    \defaultfont
    % \everymath={\displaystyle}
  }{
  \end{tcolorbox}
}

\newenvironment{subanswer}{
  \begin{tcolorbox}[%
      enlarge left by = 1.5em,%
      width = \linewidth - 1.5em,%
      breakable,%
      enhanced,%
      title=Answer,%
      title filled,%
      colframe=red!50!white,%
      colbacktitle=red!50!white,%
      attach boxed title to top left = {yshift = -3mm, xshift = 5mm},%
      colback=red!1%
    ]
    % \fontfamily{Xcharter}\selectfont
    \vspace{0.5em}
    % \fontfamily{cmr}\selectfont
    \defaultfont
    % \everymath={\displaystyle}
  }{
  \end{tcolorbox}
}

\newenvironment{subsubanswer}{
  \begin{tcolorbox}[%
      enlarge left by = 3em,%
      width = \linewidth - 3em,
      breakable,%
      enhanced,%
      title=Answer,%
      title filled,%
      colframe=red!50!white,%
      colbacktitle=red!50!white,%
      attach boxed title to top left = {yshift = -3mm, xshift = 5mm},%
      colback=red!1%
    ]
    % \fontfamily{Xcharter}\selectfont
    \vspace{0.5em}
    % \fontfamily{cmr}\selectfont
    \defaultfont
    % \everymath={\displaystyle}
  }{
  \end{tcolorbox}
}

%------------------------------------------------
% Command to print an assignment section title to split an assignment into major parts
\newcommand{\assignmentSection}[1]{ %
  { %
    \centering % Centre the section title
    \vspace{2\baselineskip} % Whitespace before the entire section title
    \rule{0.8\textwidth}{0.5pt} % Horizontal rule
    \vspace{0.75\baselineskip} % Whitespace before the section title
    {\LARGE \MakeUppercase{#1}} % Section title, forced to be uppercase
    \rule{0.8\textwidth}{0.5pt} % Horizontal rule
    \vspace{\baselineskip} % Whitespace after the entire section title
  } %
}

%----------------------------------------------------------------------------------------
%  TITLE PAGE
%----------------------------------------------------------------------------------------
\author{\textbf{\assignmentAuthorName} \ifdef{\studentID}{\\[1em] \studentID}{}} % Set the default title page author field
\date{} % Don't use the default title page date field

\title{
  \thispagestyle{empty} % Suppress headers and footers
  \vspace{0.2\textheight} % Whitespace before the title
  \textbf{\assignmentClass:\ \assignmentTitle}\\[-4pt]
  \ifdef{\assignmentDueDate}{{\small Due\ on\ \assignmentDueDate}\\}{} % If a due date is supplied, output it
  \ifdef{\assignmentClassInstructor}{{\large \textit{\assignmentClassInstructor}}}{} % If an instructor is supplied, output it
  \vspace{0.32\textheight} % Whitespace before the author name
}

%----------------------------------------------------------------------------------------
%  WARNING TEXT ENVIRONMENT
%----------------------------------------------------------------------------------------
\usepackage[framemethod=tikz]{mdframed}

\mdfdefinestyle{warning}{ %
  topline=false, bottomline=false, %
  leftline=false, rightline=false, %
  nobreak, %
  singleextra={ %
    \draw(P-|O)++(-0.5em,0)node(tmp1){}; % ChkTeX 36
    \draw(P-|O)++(0.5em,0)node(tmp2){}; % ChkTeX 36
    \fill[black,rotate around={45:(P-|O)}](tmp1)rectangle(tmp2); % ChkTeX 36
    \node at(P-|O){\color{white}\scriptsize\bf!}; % ChkTeX 36 ChkTeX 1
    \draw[very thick](P-|O)++(0,-1em)--(O);% --(O-|P); % ChkTeX 36
  }
}

% Define a custom environment for warning text
% Set the default warning to "Warning:"
\newenvironment{warn}[1][Warning:]{ %
  \medskip %
  \begin{mdframed}[style=warning] %
    \noindent{\textbf{#1}} %
  }{ %
  \end{mdframed} %
}

%----------------------------------------------------------------------------------------
%  INFORMATION ENVIRONMENT
%----------------------------------------------------------------------------------------
\mdfdefinestyle{info}{%
  topline=false, bottomline=false,
  leftline=false, rightline=false,
  nobreak,
  singleextra={%
    % Redundant two temp nodes to make the left align consistent with the warning environment
    \draw(P-|O)++(-0.5em,0)node(tmp1){}; % ChkTeX 36
    \draw(P-|O)++(0.5em,0)node(tmp2){}; % ChkTeX 36
    \fill[black](P-|O)circle[radius=0.4em]; % ChkTeX 36
    \node at(P-|O){\color{white}\scriptsize\bf i}; % ChkTeX 1 % ChkTeX 36
    \draw[very thick](P-|O)++(0,-0.8em)--(O);%--(O-|P); % ChkTeX 36
  }
}

% Define a custom environment for information
% Set the default title to "Info:"
\newenvironment{info}[1][Info:]{ %
  \medskip %
  \begin{mdframed}[style=info] %
    \noindent{\textbf{#1}} %
  }{ %
  \end{mdframed} %
}

%----------------------------------------------------------------------------------------
%  SELF_DEFINED COMMANDS
%----------------------------------------------------------------------------------------
